\documentclass[aspectratio=43]{beamer}
% \documentclass[aspectratio=169]{beamer}

% Title --------------------------------------------
\title[Lecture 4: Causal methods]{\Large Causal methods with observational data}
\author[]{Francisco Villamil}
\date[]{Research Design for Social Sciences\\MA Computational Social Science, UC3M\\Fall 2023}

\input{../beamer_preamble.tex}

\begin{document}
% ====================================================

% ----------------------------------------------------
\begin{frame}
  \titlepage
\end{frame}
% ----------------------------------------------------

\section{Intro and overview}

% ----------------------------------------------------
\begin{frame}
\frametitle{Re-cap}
\centering

\begin{itemize}
  \item[1.] Problem/topic
  \item[2.] Stories, arguments about mechanisms
  \item[3.] Research question
  \item[4.] Proper theory, concepts and operationalization
  \item[5.] Measurement, unit of analyses, data sources, etc
  \item[6.] \BGyellow{Inference strategy}
  \item[7.] Results \& interpretation
\end{itemize}

\end{frame}
% ----------------------------------------------------

% ----------------------------------------------------
\begin{frame}
\frametitle{Methods and causal inference}
\centering

\begin{itemize}
  \item Most of the time is \textbf{impossible} to control for all relevant variables (i.e. not able to close all back-door paths)
  \item So what do we do? We try to find ways to control for unobserved confounding
  \item One option is to rely on additional controlling techniques
  \begin{itemize}
    \item \BGyellow{Matching}: also depends on observables, but parametric advantages, etc (anyway, not a solution for $U$)
    \item \BGyellow{Fixed effects}: can control for group-level unobservables, and
  \end{itemize}



  \item But there are other methods often related to causal inference, not because they uncover causal relationships, but because they allow you to exploit \BGyellow{`typical' exogenous sources of variation}
  \item {\small (In H-K's \textit{The Effect}, they're called `template causal diagrams')}
\end{itemize}

\end{frame}
% ----------------------------------------------------

% ----------------------------------------------------
\begin{frame}
\frametitle{Methods and causal inference}
\centering

\begin{itemize}
  \item This is \textit{something} that introduces variation in the treatment that is independent from confounders and you can exploit to analyse (so, like randomization in an experiment). For example:
  \item[1.] Time
  \begin{itemize}
    \item we can exploit changes over time in treated vs control units, e.g. imagine checking effect of good nutrition on a child who is growing anyway
  \end{itemize}
  \item[2.] Cut-offs
  \begin{itemize}
    \item sometimes something happens just when you cross a cut-off (getting into university, winning an election, a geographical border, being born Jan 1st...)
  \end{itemize}
  \item[3.] A third, unrelated variable
  \begin{itemize}
    \item you win the lottery, you get a sudden increase in disposable income
  \end{itemize}
\end{itemize}

\end{frame}
% ----------------------------------------------------


\section{Re-cap}

% ----------------------------------------------------
\begin{frame}
\frametitle{Re-cap and final essay}
\centering

\begin{itemize}
  \item \textbf{Groups?} Send me an email
\end{itemize}

\end{frame}
% ----------------------------------------------------

% ====================================================
\end{document}
